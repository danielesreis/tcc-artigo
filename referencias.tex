\newpage

\section{Referências}


\noindent Abarra, Maja Sierhine J., et al. "Determination of Fruit Ripeness Degree of ‘Carabao’Mango (Mangifera indica L.) using Digital Photometry." Philippine Journal of Science 147.2 (2018): 249-253.
\\

\noindent Anuário brasileiro da fruticultura 2018 / Benno Bernardo Kist... [et al.]. – Santa Cruz do Sul : Editora Gazeta Santa Cruz, 2018. 88 p. Disponível em: \\ http://www.editoragazeta.com.br/sitewp/wp-content/uploads/2018/04/ \\
FRUTICULTURA\_2018\_dupla.pdf.
\\

\noindent Chagas, César D. S, et al. "Spatial prediction of soil surface texture in a semiarid region using random forest and multiple linear regressions." Catena 139 (2016): 232-240.
\\

\noindent Costa, J. D. S., Almeida, F. D. C., Figueiredo Neto, A., Cavalcante, I. H. L. Physical and mechanical parameters correlated to the ripening of mangoes (Mangifera indica L.) cv. 'Tommy Atkins', Acta Agronómica, 66 (2) (2017), pp. 186-192. \\
http://dx.doi.org/10.15446/acag.v66n2.54757.
\\

\noindent Friedman, Jerome, Trevor Hastie, and Robert Tibshirani. The elements of statistical learning. Vol. 1. No. 10. New York: Springer series in statistics, 2001.
\\

\noindent Goulart, C., Loy, F. S., Galarça, S. P., Giovanaz, M. A., Malgarim, M. B., Fachinello, J. C. Evolução do índice DA e coloração da epiderme de mangas da cultivar Tommy Atkins. Revista Iberoamericana de Tecnología Postcosecha, 14 (2013), pp. 8-13. Disponível em: http://www.redalyc.org/html/813/81327871003/index.html.
\\

\noindent Granitto, Pablo M. et al. Recursive feature elimination with random forest for PTR-MS analysis of agroindustrial products. Chemometrics and Intelligent Laboratory Systems, v. 83, n. 2, p. 83-90, 2006.
\\

\noindent Guo, P.T., Li, M.F., Luo, W., Tang, Q.F., Liu, Z.W., Lin, Z.M., 2015. Digital mapping of soil organic matter for rubber plantation at regional scale: an application of random forest
plus residuals kriging approach. Geoderma 237–238, 49–59.
\\

\noindent Hortifruti Brasil. Anuário 2017 – 2018. Disponível em: <
www.hfbrasil.org.br/br/revista/ \\
acessar/completo/anuario-2017-2018.aspx. Acesso em: 14 fev. 2019.
\\

\noindent Khairunniza-Bejo, Siti, and Syahidah Kamarudin. "Chokanan mango sweetness determination using hsb color space." 2011 Third International Conference on Computational Intelligence, Modelling & Simulation. IEEE, 2011.
\\

\noindent Li, J. B., Huang, W. Q., Zhao, C. J. Machine vision technology for detecting the external defects of fruits. Imaging Science Journal, 63 (5) (2015), pp. 241-251. doi: 10.1179/1743131X14Y.0000000088.
\\

\noindent Lima Alves, Elis Dener, and Francisco Arthur Silva Vecchia. "Análise de diferentes métodos de interpolação para a precipitação pluvial no Estado de Goiás." Acta Scientiarum. Human and Social Sciences 33.2 (2011).
\\

\noindent Mattoo, A. K.; Modi, V. V. Palmitic acid activation of peroxidase and its possible significance in mango ripening. Biochimica et Biophysica Acta (BBA)-Enzymology, v. 397, n. 2, p. 318-330, 1975.
\\

\noindent Menze, Bjoern H. et al. A comparison of random forest and its Gini importance with standard chemometric methods for the feature selection and classification of spectral data. BMC bioinformatics, v. 10, n. 1, p. 213, 2009.
\\

\noindent Nagle, M., Intani, K., Romano, G., Mahayothee, B., Sardsud, V., Müllher, J. Determination of surface color of ‘all yellow’ mango cultivars using computer vision. International Journal of Agricultural and Biological Engineering, 9 (1) (2016), pp. 42-50. Disponível em: https://ijabe.org/index.php/ijabe/article/view/1861/pdf.
\\

\noindent Nandi, Chandra Sekhar, Bipan Tudu, and Chiranjib Koley. "A machine vision-based maturity prediction system for sorting of harvested mangoes." IEEE Transactions on Instrumentation and measurement 63.7 (2014): 1722-1730.
\\

\noindent AOAC, 1997. Official Methods of Analysis of the Association of Official Analytical Chemists, sixteenth ed. Patricia Cuniff, Arlington.
\\

\noindent Pandey, Rashmi, Nikunj Gamit, and Sapan Naik. "Non-destructive quality grading of mango (Mangifera Indica L) based on CIELab colour model and size." 2014 IEEE International Conference on Advanced Communications, Control and Computing Technologies. IEEE, 2014.
\\

\noindent Salunkhe, Rahul Pralhad, and Aniket Anil Patil. "Image processing for mango ripening stage detection: RGB and HSV method." 2015 Third International Conference on Image Information Processing (ICIIP). IEEE, 2015.
\\

\noindent Teoh, C. C., and AR Mohd Syaifudin. "Image processing and analysis techniques for estimating weight of Chokanan mangoes." Journal of Tropical Agriculture and Food Science 35.1 (2007): 183.
\\

\noindent Trindade, D. C. G., Lima, M. A. C., Assis, J. S. Ação do 1-metilciclopropeno na conservação pós-colheita de manga 'Palmer' em diferentes estádios de maturação. Pesquisa Agropecuária Brasileira, 50 (9) (2015), pp. 753-762. http://dx.doi.org/10.1590/S0100-204X2015000900003.
\\

\noindent Tucker, G. A. Introducion. In: SEYMOUR, G. B. et al. Biochemistry of fruit ripening.
London: Chapman & Hall. 1993. Cap. 1, 255-266 p.
\\

\noindent Vélez-Rivera, Nayeli, et al. "Computer vision system applied to classification of “Manila” mangoes during ripening process." Food and bioprocess technology 7.4 (2014): 1183-1194.
\\

\noindent Wills, R. B. H. et al. Temperature. In: Postharvest phisiology, handling of fruits and
vegetables. Austrália: N. S. W. U. Press. 1981. 39-51 p.
\\

\noindent Yahaya, Ommi Kalsom Mardziah, et al. "Determining Sala mango qualities with the use of RGB images captured by a mobile phone camera." AIP Conference Proceedings. Vol. 1657. No. 1. AIP Publishing, 2015.
\\

\noindent Yossy, Emny Harna, et al. "Mango Fruit Sortation System using Neural Network and Computer Vision." Procedia computer science 116 (2017): 596-603.
\\

\noindent Zhang, Yongli; Yang, Yuhong. Cross-validation for selecting a model selection procedure. Journal of Econometrics, v. 187, n. 1, p. 95-112, 2015.
\\

\noindent Zheng, Hong, and Hongfei Lu. "A least-squares support vector machine (LS-SVM) based on fractal analysis and CIELab parameters for the detection of browning degree on mango (Mangifera indica L.)." Computers and Electronics in Agriculture 83 (2012): 47-51.
\\

\noindent Zhou, Qifeng et al. Structure damage detection based on random forest recursive feature elimination. Mechanical Systems and Signal Processing, v. 46, n. 1, p. 82-90, 2014.
\\