\section{Introdução}

A produção de manga é uma atividade de grande expressão no cenário da fruticultura brasileira. O País figura entre os maiores produtores da fruta, e de acordo com o Anuário Brasileiro de Fruticultura (2018), além de ser autossuficiente na produção, o Brasil também é o maior exportador, com 179 mil toneladas embarcadas em 2017.

A maior área plantada de mangas no Brasil se encontra na região do Vale do São Francisco, cuja área passou por uma expansão devido ao avanço da produção e exportação, aumentando de 27,17 ha em 2017 para 30,30 ha em 2018 (Hortifruti Brasil, 2018). Dentre as variedades produzidas na região, a ‘Palmer’ tem ganhado espaço em decorrência de novos plantios e, também, dá sobre-enxertia em plantios da variedade ‘Tommy Atkins’ (Trindade, Lima, \& Assis, 2015).

Tendo o mercado externo como um dos principias consumidores da manga e ao mesmo tempo um mercado exigente e competitivo, fazem-se necessários estudos sobre o processo de maturação da manga até a colheita, visto que o fruto colhido em estádio imaturo não será capaz de alcançar o nível de qualidade aceitável para o consumidor, podendo limitar também a conservação pós-colheita (Costa et al., 2017).

Os métodos tradicionais e mais utilizados na determinação da maturação e qualidade de frutos baseiam-se em processos destrutivos. Neste sentido, o desenvolvimento e estudo de técnicas alternativas que permitam a determinação de atributos de qualidade, de forma precisa e não invasiva, são de extrema importância (Goulart et al., 2013), principalmente no sentido de reduzir perdas quantitativas e qualitativas de produção.

O potencial de técnicas não destrutivas como ferramentas de avaliação e classificação de frutas vem sendo alvo de diferentes estudos. Modalidades de imagem são investigadas para a avaliação da qualidade, desde imagens do infravermelho próximo (NIR), à imagens multi e hiperespectrais, imagem de reflexão de iluminação estruturada, imagens visíveis à base de luz monocromática ou preto / branco até imagens em cores ou RGB (vermelho, verde e azul) (Li, Huang, \& Zhao, 2015). Com a análise e processamento de imagens digitais é possível avaliar a mudança dos aspectos visuais dos frutos de forma objetiva, integral e representativa, assim como correlacionar com atributos físico-químicos da polpa (Nagle et al., 2016).

Estudos com mangas de diferentes cultivares utilizando imagem vêm sendo desenvolvidos, no entanto, diferenças entre eles e também no ambiente de cultivo podem afetar o desempenho e consistência dos índices de maturação, tanto que atualmente não há consenso sobre o índice ideal para a manga, no caso específico para variedade ‘Palmer’, que tem lugar de destaque no cenário nacional e internacional. Assim, a escolha de técnicas de pré-processamento de imagens e variáveis empregadas devem ser testados para cada variedade, assim como a técnica de inferência empregada, em que, se um grande conjunto de amostras for utilizado, aumenta-se a robustez do processo de predição (Pereira et al., 2017).

Esses estudos ainda são incipientes no Brasil, bem como na região do Vale do São Francisco, principal pólo de produção de mangas no País, demonstrando serem ferramentas novas e promissoras para determinação de atributos de qualidade desses frutos no campo. Assim, objetivou-se com este trabalho avaliar o uso de imagens de reflectância para identificação das variáveis que devem ser extraídas para a predição de atributos de qualidade de mangas ‘Palmer’ em diferentes estádios de maturação.

\subsection{Trabalhos Relacionados}
A seguir é discorrido sobre dez diferentes trabalhos que empregam Visão computacional para determinação de atributos de qualidade em mangas de diversas variedades.

Os autores Teoh e Syaifudin (2007) determinaram, em seus trabalhos, o peso de mangas da variedade Chokanan através de uma Regressão linear. Eles inicialmente utilizaram o filtro da mediana para correção de inconsistências nas imagens, seguida da segmentação. A partir das imagens segmentadas, foi obtido o número de pixels correspondentes à manga. Utilizando esta variável de entrada em uma Regressão linear, eles obtiveram um coeficiente de correlação igual a 0,9769 e erro médio igual a 3,76\%.

Para prever sólidos solúveis totais (SST) em mangas Chokanan, Khairunniza-Bejo e Kamarudin (2011) utilizaram o espaço de cores HSB (matiz, saturação e brilho). Os autores não realizaram pré-processamento e extraíram os valores médios de matiz, saturação e brilho a partir de uma região central na manga. Eles construíram modelos de Regressão linear para as três variáveis de entrada separadamente, visando determinar o melhor canal para a predição de SST. Eles obtiveram então um coeficiente de correlação igual a -0,92 com a matiz. Para mangas possuindo SST entre as faixas 4-8, 8-13 e 13-17 ºBrix, respectivamente, eles obtiveram valores de raiz do erro quadrático médio (RMSE) iguais a 0,06, 0,02 e 0,03 ºBrix.

Zheng e Lu (2012) classificaram mangas quanto ao seu estádio de escurecimento através da LS-SVM (Máquina de vetores de suporte por mínimos quadrados). As imagens dos frutos foram pré-processadas pela subtração das mesmas pelo fundo. Os autores então extraíram os valores médios do canal L*a*b* e as variáveis fractais \textit{Box Counting Dimension}, \textit{Correlation Dimension} e \textit{Dilation Dimension}. Os autores construíram 3 classificadores: um para as variáveis de cor, outra para as variáveis fractais e outro que utilizou ambos subconjuntos. Zheng e Lu (2012) obtiveram 100\% de acurácia ao empregar todas as variáveis.

Para predição de dias restantes até o apodrecimento de mangas, Nandi et al. (2014) empregaram como pré-processamento o filtro \textit{deblurring} de Wiener e filtro da mediana. Após isto, foram extraídas as médias das intensidades RGB na manga inteira e nas regiões do cume, equador e haste, assim como a diferença destas médias e o gradiente ao longo do eixo longitudinal. As melhores variáveis de entradas foram determinadas através do algoritmo de eliminação recursiva (RFE) e SVM (Máquina de vetores de suporte). Os autores obtiveram uma acurácia média igual a 96\% ao empregar, predominantemente, variáveis derivadas do canal R.

No trabalho de Vélez-Rivera et al. (2014), foi estimado o estádio de maturação de mangas da variedade Manila, utilizando como variáveis de entrada a acidez titulável, SST, firmeza, índice RPI (\textit{Ripening Index}) e média dos pixels nos espaços de cores L*a*b* e HSB, totalizando dez variáveis. Nenhuma técnica de pré-processamento foi empregada; ao invés disso, os autores extraíram a faixa central ao longo do comprimento da manga para cada lado da mesma. Vélez-Rivera et al. (2014) empregaram então a técnica PCA (Análise de componentes principais) para determinar as variáveis mais significantes e MDA (\textit{Multiple Discriminant Analysis}) para classificação. Os autores obtiveram como melhor resultado uma acurácia igual a 100\% ao empregar todas as variáveis físico-químicas e as médias nos canais R, G, a* e b*.

Mangas das variedades Totapuri, Badami e Neelam foram utilizadas por Pandey et al. (2014), que previram a presença de doenças nas frutas e o tamanho delas. Os autores converteram as imagens para a escala de cinza, reduziram seus tamanhos e as pré-processaram com filtro da mediana e uma técnica de aguçamento. As imagens foram então segmentadas e convertidas para o espaço L*a*b*, em que b* foi utilizado para determinar o limiar entre as mangas saudáveis e com doenças. O tamanho das mangas foi previsto através de um sistema de inferência \textit{Fuzzy}, possuindo como entrada a área estimada e o diâmetro. Os autores obtiveram uma acurácia média igual a 93,33\% para determinação da saúde das mangas e 91,41\% para classificação quanto ao tamanho.

Yahaya et al. (2015) determinaram os atributos SST, acidez titulável e firmeza em mangas da variedade Sala, extraindo os valores médios no espaço RGB e utilizando comotécnica de inferência a MLR (\textit{Multiple Linear Regression}). Nenhuma técnica de pré-processamento foi mencionada. Eles obtiveram os coeficientes de correlação 0,875, 0,814 e 0,913 para a firmeza, SST e acidez titulável respectivamente, enquanto que os valores de RMSE foram iguais a 1,392 kgf, 1,218 ºBrix e 0,166 pH.

Os estádios de maturação de mangas Alphonso foram determinadas através dos espaços de cores RGB e HSV (matiz, saturação e valor) por Salunkhe et al. (2015). O único tratamento realizado nas imagens foi a segmentação e as variáveis extraídas foram as médias das intensidades, assim como as taxas R/G, R/B e S/H. A partir da classificação manual das mangas, foram determinados os valores das três taxas de forma que as classes fossem discriminadas sem erro. A partir dos limiares obtidos para as três taxas, foi desenvolvido um algoritmo para classificação das mangas, através de comandos \textit{if-else}. Para o modelo com apenas o canal RGB, foi obtida uma acurácia igual a 90,4\%, taxa de falsos positivos igual a 2,57\% e taxa de verdadeiros positivos igual a 89,77\%. 

Yossy et al. (2017) classificaram mangas da variedade Gincu quanto ao estádio de maturação e tamanho. As imagens foram pré-processadas través das operações morfológicas de abertura e fechamento e, posteriormente, reduzidas para o tamanho 16x16 pixels. Elas foram então convertidas do espaço RGB para o HSV, de forma a determinar a cor dominante da manga. O \textit{array} resultante, de 257 posições (256 pixels da imagem mais a cor dominante), foi utilizado como entrada em uma rede neural com função e ativação do tipo sigmoide e algoritmo \textit{back propagation}. A acurácia obtida foi igual a 94\%.

Em estudos com mangas da variedade Carabao, Abarra et al. (2018) determinaram os atributos acidez titulável, açúcares totais, amido total, firmeza, acidez titulável, SST e total de açúcares reduzido, sem o emprego de pré-processamento nas imagens. As variáveis extraídas consistiram nos valores médios das intensidades dos pixels nos espaços RGB, HSV e L*a*b*, sendo utilizadas como entrada em modelos de regressão linear para cada atributo de qualidade. Os melhores resultados alcançados foram para acidez titulável e firmeza, ao utilizar apenas o canal L*, obtendo coeficientes de correlação iguais a 0,977 e 0,968 respectivamente. 

Na Tabela \ref{tab:artigos_att}, é feita uma sumarização das variáveis extraídas pelos autores, atributos de qualidade estimados e variedades empregadas.

\begin{center}
    \begin{table}[H]
    \setlength{\tabcolsep}{2.5pt}
    \tiny
    \caption{\label{tab:artigos_att} Sumarização dos trabalhos relacionados.}
        \begin{tabular}{>{\centering}m{2cm} >{\centering}m{1.5cm} >{\centering}m{2cm} >{\centering}m{0.7cm} >{\centering}m{1.4cm} >{\centering}m{1cm} >{\centering}m{0.7cm} >{\centering}m{1cm} >{\centering}m{0.7cm} >{\centering}m{1cm} >{\centering}m{1cm}cccccccccc}
        \hline
        Autores & Atributo alvo & Variedade & Média RGB & Diferença de médias e gradiente RGB & R/G, R/B e S/H & Média HSV & Cor HSV dominante & Média L*a*b* & Número de pixels & Variáveis fractais & Diâmetro    \\ \hline
        Teoh e Syaifudin (2007)                  & Massa                    & Chokanan &   &   &   &   &   &   & X &   &   \\ \hline
        Khairunniza-Bejo e Kamarudin (2011)      & SST                      & Chokanan &   &   &   & X &   &   &   &   &   \\ \hline  
        Zheng e Lu (2012)                        & Apodrecimento            & Sannianmang &   &   &   &   &   & X &   & X &   \\ \hline  
        Nandi et al. (2014)               & Apodrecimento            & Kumrapali, Amrapali, Sori, Langra, Himsagar & X & X &   &   &   &   &   &   &   \\ \hline 
        Vélez-Rivera et al. (2014)                      & Maturação                & Manila &   &   &   & X &   & X &   &   &   \\ \hline  
        Pandey et al. (2014)                            & Tamanho e doença         & Totapuri, Badami e Neelam &   &   &   &   &   & X & X &   & X \\ \hline   
        Yahaya et al.(2015)                            & SST, acidez titulável e firmeza    & Sala & X &   &   &   &   &   &   &   &   \\ \hline  
        Salunkhe et al. (2015)                          & Maturação                & Alphonso & X &   & X & X &   &   &   &   &   \\ \hline  
        Yossy et al. (2017)                             & Maturação                & Gincu &   &   &   &   & X &   &   &   &   \\ \hline   
        Abarra et al. (2018)                            & SST, acidez titulável e firmeza    & Carabao & X &   &   & X &   & X &   &   &   \\ \hline
        \end{tabular}
    \end{table}
\end{center}

A partir da Tabela acima, nota-se que não há uma unanimidade ou padrão quanto às variáveis escolhidas. Para os atributos alvo SST, firmeza e acidez titulável, foram utilizadas apenas variáveis baseadas em cor. A partir dos espaços de cores testados, os autores selecionaram o canal que garantiu o melhor resultado. Por outro lado, para determinação da massa, apenas o número de pixels correspondente à manga é utilizado como entrada em um modelo. 

Ademais, nota-se que dentre as diversas variáveis possíveis, os autores que estimaram estes atributos de qualidade limitaram-se a utilizar apenas uma variável. Assim, com este trabalho busca-se verificar se com a utilização de mais informações visuais da manga é possível obter resultados superiores para a determinação de massa, SST, firmeza e acidez titulável. 

Por fim, nota-se que em nenhum artigo a variedade 'Palmer' foi utilizada. Dessa forma, espera-se que com esse trabalho seja estabelecida a abordagem ideal para predição não destrutiva de atributos de qualidade dessa variedade, contribuindo assim para o desenvolvimento das regiões produtoras do fruto, como o Vale do São Francisco.